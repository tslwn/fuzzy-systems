\renewcommand{\thesection}{Q\arabic{section}}

\section{Fuzzy control and sound synthesis}

\renewcommand{\thesection}{\arabic{section}}
\renewcommand{\thesubsection}{\arabic{subsection}}

\subsection{Introduction}
\label{q1:introduction}

Natural and artificial systems are frequently complex, non-linear, and operate under
variable conditions.
Accordingly, approaches to modelling and controlling these systems must be designed to
approximate non-linear functions and handle uncertainty.
Fuzzy systems and neural networks are different, and sometimes complementary,
approaches to modelling and control.
The learned parameters of neural networks are often difficult to interpret, and their
behaviours are difficult to explain.
By contrast, fuzzy systems can learn linguistically interpretable rules that can be
verified by humans, or explicitly encode human knowledge \parencite[1593]{Babuska1996}.
Interpretability and explainability are central concerns in artificial-intelligence
research \parencite[e.g.][]{Gilpin2018}, particularly in its applications to
risk-averse domains, such as healthcare, robotics, and industrial processes.
Additionally, in creative applications, the ability to encode human knowledge and
define novel behaviours is a desirable feature of fuzzy approaches.
In this essay, I provide a brief overview of fuzzy control systems in
\cref{q1:fuzzy-control} and an example of its application to sound synthesis in
\cref{q1:sound-synthesis}.
Finally, I contrast this case study with neural-network approaches in
\cref{q1:neural-networks}.

\subsection{Fuzzy control}
\label{q1:fuzzy-control}

\begin{figure}
  \centering
  \begin{tikzpicture}[inner sep=0.15in, node distance=0.5in, line width=0.01in]
    % Object
    \node[draw] (obje) {Object};

    \node[above=of obje] (ext) {};
    \draw[-Latex] (ext) -- (obje);

    \node[coordinate, right=0.25in of obje] (outa) {};
    \node[coordinate, right=0.25in of outa] (outb) {};
    \draw[-] (obje) -- (outa);
    \draw[-Latex] (outa) -- (outb);

    % Controller
    \node[draw, left=1.25in of obje] (ctrl) {Controller};
    \draw[-Latex] (ctrl) -- node[above] {Actions} (obje);

    \node[coordinate, left=of ctrl] (refe) {};
    \draw[-Latex] (refe) -- (ctrl);

    % Sensor
    \node[draw, below left=0.375in and 0in of obje] (sens) {Sensor};
    \draw[-Latex] (outa) |- (sens);
    \draw[-Latex] (sens) -| node[below] {Conditions} (ctrl);

    \node[below=of sens] (nois) {};
    \draw[-Latex] (nois) -- (sens);
  \end{tikzpicture}
  \caption{A basic feedback control system \parencite[27]{Doyle1990}.}
  \label{fig:feedback-control}
\end{figure}

Since Zadeh's introduction of fuzzy sets \parencite*{Zadeh1965}, fuzzy logic has been
widely applied to control systems \parencite[330]{Klir1995}.
A basic feedback control system comprises a controlled object, sensors that measure the
conditions of the object, and a controller that generates actions to apply to it
\parencite[27]{Doyle1990}.
The controller applies inference rules to the conditions to generate actions.
A schematic of this arrangement is given in \cref{fig:feedback-control}.
Generally, sensors produce crisp measurement values and actuators apply actions that
are defined by crisp values.
Hence, to apply fuzzy inference rules, a fuzzy controller:
\begin{enumerate}
  \item fuzzifies the conditions of the object, i.e., converts them to fuzzy sets;
  \item applies fuzzy inference rules to the fuzzified conditions; and
  \item defuzzifies the outputs of the inference rules, i.e., converts them to crisp
        values.
\end{enumerate}
This procedure is depicted in \cref{fig:fuzzy-control}, after
\textcite[331-332]{Klir1995}.

\begin{figure}
  \centering
  \begin{tikzpicture}[inner sep=0.15in, line width=0.01in]
    % Object
    \node[draw] (obje) {Object};

    % Fuzzification
    \node[draw, minimum width=1.5in, below right=0.375in and 0.5in of obje] (fuzz) {Fuzzification};
    \draw[-Latex] (obje) |- node[below] {Conditions} (fuzz);

    % Defuzzification
    \node[draw, minimum width=1.5in, above right=0.375in and 0.5in of obje] (defu) {Defuzzification};
    \draw[-Latex] (defu) -| node[above] {Actions} (obje);

    % Inference
    \node[draw, right=2in of obje] (infe) {Inference rules};
    \draw[-Latex] (fuzz) -| (infe);
    \draw[-Latex] (infe) |- (defu);

    % Controller
    \node[draw, dashed, inner sep=0.25in, label=above:Fuzzy controller, fit=(defu) (fuzz) (defu) (infe)] {};
  \end{tikzpicture}
  \caption{A fuzzy control system \parencite[331-332]{Klir1995}.}
  \label{fig:fuzzy-control}
\end{figure}

\subsubsection{Fuzzification}
\label{q1:fuzzification}

In a fuzzy system, variables are described by fuzzy sets \parencite[327]{Klir1995}.
To convert a crisp value to a fuzzy set, the `universe of discourse' of the variable,
i.e., the range of values it takes, and the semantics of the fuzzy set that describes
it, i.e., its membership function, must be defined \parencite[60]{Sugeno1985}.
For example, instead of a crisp value of $60\,\text{dB}$, the amplitude of a sound may
be expressed in terms of the membership values of the elements of a fuzzy set
$\{\text{low}, \text{medium}, \text{high}\}$.
In this way, a fuzzy representation of a variable can account for the inherent
uncertainty of measurement values and the limited resolution of measurement
instruments, and present a more intuitive description to human designers and operators.

\subsubsection{Inference}
\label{q1:inference}

Zadeh later presented an approach to fuzzy system modelling based on \emph{linguistic
  variables}, i.e., variables whose values are expressions in a natural or artificial
language \parencite*[199]{Zadeh1975}.
As above, the amplitude of a sound event could be described as `low', `medium', or
`high' instead of a crisp value in decibels.
\textcite{Mamdani1975} were the first to apply this approach to control systems.
In this scheme, the inference rules of a control system are \emph{if-then} statements
whose premises and consequences are expressed in terms of linguistic variables
\parencite[57-58]{Nguyen2019}.
For example, the application of the rule `\emph{if} the amplitude is low \emph{then}
increase the amplitude' to a sound event whose amplitude is `low' may produce an event
whose amplitude is `medium'.
An \emph{if-then} statement is an example of \emph{modus ponens}, an argumentative form
in propositional logic -- Zadeh's approach is thus sometimes called generalized modus
ponens \parencite[e.g.][]{Dubois1984}.

A key advantage of this type of system is that it is generally easier for humans to
formulate inference rules in linguistic terms than in mathematical terms, which helps
to encode expert knowledge in the system and to understand its behaviour.
On the other hand, Mamdani controllers do not construct an explicit model of the
controlled process, which makes it difficult to analyse its stability
\parencite[58]{Nguyen2019}.
Stability analysis is an important aspect of the design of control systems, which are
frequently used to maintain an equilibrium state \parencite[3]{Doyle1990}.

However, not all fuzzy control systems are based on linguistic variables.
Accordingly, \textcite{Sugeno1985} classifies fuzzy systems according to the form of
the consequences of their inference rules (\cref{tab:types-of-fuzzy-systems}).
For example, Takagi-Sugeno controllers map input values to (usually) linear functions
\parencites{Takagi1985}[58-59]{Nguyen2019}.
This approach is particularly useful for modelling non-linear systems, but sacrifices
the interpretability of linguistic variables in its outputs.
\textcite{Nguyen2017} presents a third type of fuzzy system, in which the consequences
of the inference rules are real numbers.
These `singleton' or piecewise multi-affine systems are computationally inexpensive and
amenable to stability analysis \parencite[63-64]{Nguyen2019}.

\begin{table}
  \centering
  \begin{tabular}{lll}
    \toprule
    Type          & Consequences       & Reference
    \\
    \midrule
    Mamdani       & fuzzy sets         & \textcite{Mamdani1975}
    \\
    Takagi-Sugeno & (linear) functions & \textcite{Takagi1985}
    \\
    Singleton     & real numbers       & \textcite{Nguyen2017}
    \\
    \bottomrule
  \end{tabular}
  \caption{Types of fuzzy systems \parencite{Nguyen2019}.}
  \label{tab:types-of-fuzzy-systems}
\end{table}

\subsubsection{Defuzzification}
\label{q1:defuzzification}

The output fuzzy sets of a fuzzy controller must be converted to crisp values to be
applied to the controlled object.
Generally, approaches to defuzzification are based on either the maxima of the
membership function of the fuzzy set, its distribution, or the area under its curve,
depending on the type of the variable \parencite[166-172]{Leekwijck1999}.
The first two of these approaches are illustrated by
definitions~\ref{def:random-choice-of-maxima} and \ref{def:centre-of-gravity}, in which
$\tilde{A}$ is a fuzzy set with membership function $\chi_A : W \to \left[0, 1\right]$.

\begin{definition}
  \label{def:random-choice-of-maxima}
  A \emph{random choice of maxima} is the result of an experiment with the
  probability distribution:
  \begin{equation}
    P(x) =
    \begin{cases}
      \lvert \text{core}(\tilde{A}) \rvert^{-1} & \text{if } x \in \text{core}(\tilde{A})
      \\
      0                                         & \text{otherwise}
    \end{cases}
    \ ,\quad
    \text{core}(\tilde{A}) = \left\{x \in W \mid \chi_A(x) = \max_{y \in W} \chi_A(y)\right\}
  \end{equation}
\end{definition}

\begin{definition}
  \label{def:centre-of-gravity}
  The \emph{centre of gravity} of $\tilde{A}$ is:
  \begin{equation}
    \text{cog}(\tilde{A}) = \frac{\sum_{x \in W} x \cdot \chi_A(x)}{\sum_{x \in W} \chi_A(x)}.
  \end{equation}
  If $\chi_A$ is a probability distribution, then the centre of gravity is equivalent to
  the expected value of the random variable $X \in W$ where $P(x) = \chi_A(x)$.
\end{definition}

\subsection{Case study: sound synthesis}
\label{q1:sound-synthesis}

The opportunity for humans to define the inference rules that operate a control system
has clear benefits for creative applications.
As \textcite[1-2]{Cadiz2020} explains, musical concepts are frequently imprecise, such
as the directives of tempo and dynamics in a musical score.
Furthermore, non-linearity and the complex interplay of components are desirable
characteristics of music-making systems.\footnote{For example, the musician and
  synthesizer designer Vlad Kreimer cites these characteristics as key principles of his
  design philosophy \parencite{mylarmelodies2023}.
}
A promising creative application of fuzzy systems is sound synthesis, i.e., the
computer-based generation of sound, which is surveyed by \textcite{dePoli1983}.
In general, a synthesis algorithm involves many parameters, whose values must be
manually programmed or otherwise controlled to produce a desired sound.
Parametric control is thus an important aspect of electronic composition, performance,
and sound design.

\textcite{Cadiz2020} describes an application of fuzzy control to
parametric control of granular synthesis.
A granular synthesis algorithm generates sound from many very short sound events or
`grains' \parencite{Roads1988}, taking inspiration from physics \parencite{Gabor1946}.
It is particularly amenable to novel control methods due to the large number of input
parameters and the opacity of their relations to the output sound
\parencite[11]{Cadiz2020}.
In this instance, the controlled object is a \texttt{Max/MSP}\footnote{See
  \href{https://cycling74.com/products/max}{\texttt{https://cycling74.com/products/max}}
  or \textcite{Manzo2011}, for example.
} object
with five parameters that are varied by the control system.
The evolution of these parameters over time according to fuzzy inference rules
generates complex time-series from a comparably small number of user inputs.
Naturally, this methodology can be applied to control other synthesis algorithms
\parencite[e.g.][]{Cadiz2018} and software environments.

\subsection{Neural networks}
\label{q1:neural-networks}

Various authors have advocated for fuzzy set theory in the context of control systems
and artificial intelligence, as opposed to probability and statistics, due to its
ability to explicitly represent different kinds of uncertainty
\parencite[249]{Laviolette1995}.
Neural networks are probabilistic models that are commonly used to approximate
non-linear functions by learning from data.
In contrast to \cref{q1:sound-synthesis}, for instance, \textcite{Bitton2020} describe
a granular sound synthesis technique based on a generative neural network.
In this case, the properties of the grains are represented by a latent space, which is
learned by a variational auto-encoder, and the network generates the waveform directly,
instead of controlling a synthesis algorithm.
This approach does not allow for the explicit definition of inference rules, but it can
learn to generate waveforms that are similar to a corpus of audio.
Closer parallels to the work of \textcite{Cadiz2020} are provided by
\textcite{Fiebrink2009}, who presented a system to learn mappings between user inputs
and algorithm parameters, and \textcite{Jonason2020}, who presented a similar
`control-synthesis' approach to transforming user inputs, based on a recurrent neural
network.
Thus, the principal difference between neural-network and fuzzy approaches to control
systems in this context is whether the inference rules are explicitly defined by the
user or learned from data.
Generally, however, fuzzy inference rules can also be constructed from data, including
by the use of a neural network \parencite[281,295-296]{Klir1995}.
